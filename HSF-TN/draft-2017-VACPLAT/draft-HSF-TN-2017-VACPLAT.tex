\documentclass[12pt,a4paper]{article}

% Variables that controls behaviour
\usepackage{ifthen} % for conditional statements
\newboolean{pdflatex}
\setboolean{pdflatex}{true} % False for eps figures 

\newboolean{inbibliography}
\setboolean{inbibliography}{false} %True once you enter the bibliography


\textheight=230mm
\textwidth=160mm
\oddsidemargin=7mm
\evensidemargin=-10mm
\topmargin=-10mm
\headsep=20mm
\columnsep=5mm
\addtolength{\belowcaptionskip}{0.5em}

\renewcommand{\textfraction}{0.01}
\renewcommand{\floatpagefraction}{0.99}
\renewcommand{\topfraction}{0.9}
\renewcommand{\bottomfraction}{0.9}

\setlength{\hoffset}{-2cm}
\setlength{\voffset}{-2cm}

% Page defaults ...
\topmargin=0.5cm
\oddsidemargin=2.5cm
\textwidth=16cm
\textheight=22cm

% Don't chase after perfection
\raggedbottom
\sloppy

\usepackage{float}
\usepackage{microtype}
\usepackage{lineno}    % Line numbering during drafting
\usepackage{xspace}    % Avoid problems with missing or double spaces after predefined symbold
\usepackage{caption}   % These three command get the figure and table captions automatically small
\renewcommand{\captionfont}{\small}
\renewcommand{\captionlabelfont}{\small}

%% Graphics
\usepackage{graphicx}  % to include figures (can also use other packages)
\usepackage{color}
\usepackage{colortbl}

%% Math
\usepackage{amsmath} % Adds a large collection of math symbols
\usepackage{amssymb}
\usepackage{amsfonts}
\usepackage{upgreek} % Adds in support for greek letters in roman typeset

%% fix to allow peaceful coexistence of line numbering and
%% mathematical objects
%% http://www.latex-community.org/forum/viewtopic.php?f=5&t=163
%%
\newcommand*\patchAmsMathEnvironmentForLineno[1]{%
\expandafter\let\csname old#1\expandafter\endcsname\csname #1\endcsname
\expandafter\let\csname oldend#1\expandafter\endcsname\csname
end#1\endcsname
 \renewenvironment{#1}%
   {\linenomath\csname old#1\endcsname}%
   {\csname oldend#1\endcsname\endlinenomath}%
}
\newcommand*\patchBothAmsMathEnvironmentsForLineno[1]{%
  \patchAmsMathEnvironmentForLineno{#1}%
  \patchAmsMathEnvironmentForLineno{#1*}%
}
\AtBeginDocument{%
\patchBothAmsMathEnvironmentsForLineno{equation}%
\patchBothAmsMathEnvironmentsForLineno{align}%
\patchBothAmsMathEnvironmentsForLineno{flalign}%
\patchBothAmsMathEnvironmentsForLineno{alignat}%
\patchBothAmsMathEnvironmentsForLineno{gather}%
\patchBothAmsMathEnvironmentsForLineno{multline}%
\patchBothAmsMathEnvironmentsForLineno{eqnarray}%
}

% Get hyperlinks to captions and in references.
% These do not work with revtex. Use "hypertext" as class option instead.
\usepackage{hyperref}    % Hyperlinks in references
\usepackage[all]{hypcap} % Internal hyperlinks to floats.

% Make this the last packages you include before the \begin{document}
\usepackage{cite} % Allows for ranges in citations
\usepackage{mciteplus}

\usepackage{longtable} % only for template

\begin{document}

\renewcommand{\thefootnote}{\fnsymbol{footnote}}
\setcounter{footnote}{1}

\begin{titlepage}
\pagenumbering{roman}


\vspace*{-1.5cm}
\centerline{\large THE HEP SOFTWARE FOUNDATION (HSF)}
\vspace*{1.5cm}
\noindent
\begin{tabular*}{\linewidth}{lc@{\extracolsep{\fill}}r@{\extracolsep{0pt}}}

\\
 & & HSF-TN-2016-04 \\  % ID 
 & & \today \\October 4, 2016 \\ % Date - Can also hardwire e.g.: 23 March 2010
 & & \\
% not in paper \hline
\end{tabular*}

\vspace*{4.0cm}

% Title --------------------------------------------------
{\bf\boldmath\huge
\begin{center}
  Vacuum Platform
\end{center}
}

\vspace*{2.0cm}

% Authors -------------------------------------------------
\begin{center}
A.~McNab$^1$
\bigskip\\
{\it\footnotesize
$ ^1$University of Manchester
}
\end{center}

\vspace{\fill}

% Abstract -----------------------------------------------
\begin{abstract}
  \noindent

This technical note describes components of the Vacuum Platform 2.0 developed by
GridPP for managing VMs, including the \texttt{\$JOBOUTPUTS}, VacQuery, and 
VacUserData interfaces. This updates the 1.0 specification explained in
HSF-TN-2016-04\cite{HSFVACPLAT1}.

\end{abstract}

\vspace*{2.0cm}

\vspace{\fill}

{\footnotesize 
\centerline{\copyright~Named authors on behalf of the HSF, licence \href{http://creativecommons.org/licenses/by/4.0/}{CC-BY-4.0}.}}
\vspace*{2mm}

\end{titlepage}

\cleardoublepage

\renewcommand{\thefootnote}{\arabic{footnote}}
\setcounter{footnote}{0}

%%%% Uncomment next 2 lines if desired
\tableofcontents
%\cleardoublepage

\pagestyle{plain} % restore page numbers for the main text
\setcounter{page}{1}
\pagenumbering{arabic}

%% Uncomment during drafting and review.
%% Comment before a final submission.
%\linenumbers

\section{Introduction}
\label{sec:Introduction}

This technical note describes components of GridPP's Vacuum Platform 2.0 for 
managing virtual machines (VMs) to run jobs for WLCG and other HEP 
experiments. It updates HSF-TN-2016-04\cite{HSFVACPLAT1} which described verison 1.0.

The \texttt{\$JOBOUTPUTS}, VacQuery, and VacUserData interfaces are described, 
which have been developed for managing the VM lifecycle. These are used 
by two GridPP software systems, Vac and Vcycle, which can be described 
as VM lifecycle managers (VMLMs).

This note is written in terms of VMs, but the interfaces have been 
designed to be generalise to other forms of logical machine in the
future, such as Docker containers and unikernels.

The term ``resource provider'' is used to refer to the entity which is 
able to take the decision about creating each VM. That is, the decision 
about whether resources will be provided or not. Typically, this is owner 
of an OpenStack or other cloud tenancy managed by Vcycle or the manager of Vac VM 
factories. The term is not used here to refer to higher or lower layers of resource
provision in terms of legal owners of services and hardware, funding
agencies, operators of the site infrastructure etc.

A location at which VMs can be created managed by one or more VMLMs
which are cooperating to achieve a set of target shares is referred to
as a ``space''. This is equivalent to a Compute Element (CE) at a Grid site,
and spaces must be given CE-style DNS names in DNS space available to the
resource provider. However, it is not necessary to register the 
space name in the corresponding DNS zone. For Vac, a space is a set
of VM factory machines which are communicating via VacQuery and may
be said to be neighbours. For Vcycle, a space corresponds to an 
OpenStack or similar tenancy or project, with a specified endpoint to 
contact and identity tokens to use.

Each space is occupied by VMs which are instances of one or more 
``machinetype'' that the VMLM is able to create. Each machinetype
corresponds to a specific combination of VM boot image and 
contextualization.

\section{Environment}
\label{sec:env}

Where possible, an approximation of OpenStack's environment for VMs, which 
is derived from EC2, should be provided. Any contexualization 
user\_data file required and a metadata service should be provided via a 
``Magic IP'' 
HTTP service at 169.254.169.254 from the point of view of the VM. 
Monolithic VM images which do not use a user\_data file require a 
metadata service to be able to discover the 
URLs of the \texttt{\$MACHINEFEATURES}, \texttt{\$JOBFEATURES}, and 
\texttt{\$JOBOUTPUTS} locations.

As not all IaaS cloud systems provide metadata services, VMs and VMLMs 
should also implement the VacUserData substitutions described 
in section~\ref{sec:vacuserdata}. These include substitutions giving the 
URLs of the \texttt{\$MACHINEFEATURES}, \texttt{\$JOBFEATURES}, and 
\texttt{\$JOBOUTPUTS} locations.

VMLMs must ensure they support VMs which use Cloud Init contextualization.

VMs must not block access to the address 169.254.169.253, which is used
by Amazon for DNS and by Vac for the default route, the local DNS, and 
for the Machine/Job Features HTTP service.

\section{Machine/Job Features}
\label{sec:mjf}

The Machine/Job Features (MJF) mechanism described in \cite{HSFMJF} allows
resource providers to communicate information to batch jobs and virtual
machines, including the number of processors they are allocated and
how long they may run for. The MJF terminology is derived from batch
job environments, and job equates to virtual machine when applied to
virtualized environments such as the Vacuum Platform. 

Resource providers using the Vacuum Platform must make the MJF
\texttt{\$MACHINEFEATURES} and \texttt{\$JOBFEATURES} locations
available over HTTP(S) rather than through a shared filesystem, and
should publish the URLs of these locations in OpenStack/EC2 
\texttt{machinefeatures} and \texttt{jobfeatures} metadata 
tags and using the VacUserData substitutions in the user\_data files
supplied to VMs.

The value of \texttt{\$JOBFEATURES/job\_id} should be set to the VM UUID by
the VMLM as soon as it is known. For example, with Vac the UUID is
chosen by VMLM and its value can be set when the first
\texttt{\$JOBFEATURES}
key/values are created. However with Vcycle managing OpenStack, the
VM UUID is only available after the VM has been created, and is then
recorded by Vcycle in the \texttt{\$JOBFEATURES} directory it provides.

\section{\$JOBOUTPUTS}
\label{sec:joboutputs}

The \texttt{\$JOBOUTPUTS} mechanism is an extension to Machine/Job Features by
which the URL of a location to which VMs can write status and log files
is communicated to the VMs. This value of the \texttt{\$JOBOUTPUTS} URL should
be given in the same way as the \texttt{\$MACHINEFEATURES} and 
\texttt{\$JOBFEATURES} URLs, using a VacUserData substitution and an
OpenStack/EC2 \texttt{joboutputs} metadata key. 

Any log file which the VMs wish to make available to resource providers
may be written to the \texttt{\$JOBOUTPUTS} location, for later examiniation
in case of problems. All of these files must have unique names, and
are all written to the same level (``directory'') of the URL
space on the \texttt{\$JOBOUTPUTS} HTTP(S) server. This mechanism is also used 
to provide the \texttt{shutdown\_message} file described in the next section.

\subsection{Shutdown Messages}
\label{sec:shutdownmessages}

When VMs finish, they should write a \texttt{shutdown\_message} file to
\texttt{\$JOBOUTPUTS/shutdown\_message} containing one line of text without
a trailing newline character. This text consists of a
three digit shutdown message code in the range 100-999, a space, 
and then a human-readable description of the message code. 

The message code (and not the human-readable description) will
be used by the resource provider's software to determine why
the VM finished and whether to create more VMs of this type in
the immediate future as slots become available.

\begin{center}
\begin{table}[H]
\label{table:shutdowncodes}
\begin{center}
\begin{tabular}{l}
\\
100 Shutdown as requested by the VM's host/hypervisor \\
200 Intended work completed ok \\
300 No more work available from task queue \\
400 Site/host/VM is currently banned/disabled from receiving more work \\
500 Problem detected with environment/VM provided by the site \\
600 Grid-wide problem with job agent or application within VM \\
700 Transient problem with job agent or application within VM \\
\\
\end{tabular}
\caption{\label{opt}Shutdown codes and messages}
\end{center}
\end{table}
\end{center}

The shutdown codes are designed to be extensible by the insertion of 
intermediate numbers for finer-grained reporting. This is similar to the
three digit response codes of internet protcols such as SMTP and HTTP.

\section{Image URLs}
\label{sec:imageurls}

Experiments should provide the HTTPS URL of the image file required to boot
their VMs, which VMLMs should use. VMLMs should support both standard CAs
and Internationl Grid Trust Federation (IGTF) endorsed CAs when verifying
the X.509 certificates used by the relevant HTTPS webserver.

To avoid overloading these webservers, VMLMs must cache images by 
Last-Modified time, and should use the HTTP If-Modified-Since mechanism
when fetching images. If this header is used, then it is acceptable to
check the URL for updates each time a VM is created.

Where the VMLM is unable to update the image used to boot the VMs itself, 
it should attempt to verify that the image being used is current and refuse
to create new VMs with an old image. Typically this applies to IaaS cloud
systems where users are unable to upload new images, or a manual upload step
is required. VMLM authors should consider how resource providers will be
made aware of this situation when it arises, but for scalability reasons,
the VMLM should not rely on the experiment suffering from VMs failing 
due to an out of date VM image and then notifying resource providers.

\section{VacUserData templates}
\label{sec:vacuserdata}

In most cases, a generic image such as CernVM is used which then
requires further contextualization as the VM starts using a user\_data
file supplied by VMLM. The VMLM must be able to retrieve a template for
the user\_data file
from an HTTPS URL nominated by the experiment each time a VM is to 
be created. That is, without any caching. The VMLM must include an
appropriate HTTP User-Agent header indicating the VMLM implementation
and version when making this request to allow experiments to monitor
which VMLM versions are in use. The VMLM should support both standard CAs
and IGTF-endorsed CAs when verifying the X.509 certificates used by the
relevant HTTPS webserver.

The VMLM must apply the following pattern
based substitutions to the user\_data template supplied by the experiment.
These patterns are all in the form \texttt{\#\#user\_data\_XXX\#\#}.

The following substitutions are performed automatically
using data the VMLM holds internally:

\begin{quote}
\begin{description}
\setlength{\parskip}{0pt}
\item{\texttt{\#\#user\_data\_space\#\#}} Space name
\item{\texttt{\#\#user\_data\_machinetype\#\#}} Name of the machinetype of this VM
\item{\texttt{\#\#user\_data\_machine\_hostname\#\#}} Hostname assigned to the VM by the VMLM
\item{\texttt{\#\#user\_data\_manager\_version\#\#}} A string giving the VMLM version
\item{\texttt{\#\#user\_data\_manager\_hostname\#\#}} Hostname of the VMLM
\item{\texttt{\#\#user\_data\_manager\_machinefeatures\_url\#\#}} \texttt{\$MACHINEFEATURES} URL (section~\ref{sec:mjf})
\item{\texttt{\#\#user\_data\_manager\_jobfeatures\_url\#\#}} \texttt{\$JOBFEATURES} URL (section~\ref{sec:mjf})
\item{\texttt{\#\#user\_data\_manager\_joboutputs\_url\#\#}} \texttt{\$JOBOUTPUTS} URL (section~\ref{sec:joboutputs})
\end{description}
\end{quote}

The VMLM must also provide a mechanism for the resource provider
to specify strings or files whose static values will be used in pattern 
substitutions required by the VM. These patterns take the form
\texttt{\#\#user\_data\_option\_XXX\#\#} where XXX is an arbitrary string
consisting of letters, numbers, and underscores. 

If the VM requires the address(es) of HTTP proxies to use with CernVM-FS,
it must expect this value as the special pattern 
\texttt{\#\#user\_data\_option\_cvmfs\_proxy\#\#} The VM must be able to
accept compound CernVM-FS proxy expressions containing semicolon and pipe 
characters. Typically this will involve placing the substitution pattern
in appropriate quotation marks within the user\_data template.

If the VM requires an X.509 proxy, it must expect that the special pattern
\texttt{\#\#user\_data\_option\_x509\_proxy\#\#} will be replaced by the 
PEM encoded X.509 certificates and RSA private key which compromise the
proxy. VMLMs should provide a mechanism for creating X.509 proxies
dynamically for each VM instance from a host or robot certificate owned
by the resource provider, with an X.509 proxy lifetime reflecting
the maximum VM lifetime. 

The VM must not assume that any other grid, HEP middleware, or scripts
are running as part of the VMLM and able to provide dynamic values for
pattern substitutions. For example, it
must not require that resource providers provide proxies with VOMS
attributes to the VM. If this is needed, the VM should use
the proxy provided to obtain the VOMS credentials itself, using software 
managed by the experiment within the VM.

\section{VacQuery}
\label{sec:vacquery}

The VacQuery protocol specifies queries and status messages
which can be sent over UDP as short JSON documents.

The principal use of the VacQuery protocol is to allow Vac factories to
gather information from their neighbours about what VMs are running for
what machinetypes. This is done using the machinetypes\_query and
machinetype\_status UDP messages. Factory and machine message pairs
are also supported which can be used for automated or manual 
monitoring of Vac-based sites.

VacQuery queries sent to Vac daemons take the form of JSON documents
in packets directed to the unused UDP port 
995.\footnote{In Roman numerals, V=5 and M=1000. 995 could be written
as VM = 1000 - 5, although this violates conventions invented in 
modern times.} Responses are sent to the UDP port from which the
query was sent. The protocol has been designed to keep JSON messages and
IP headers below the ethernet MTU of 1500 bytes to avoid fragmentation
on local networks. 

All dates/times in VacQuery messages are expressed as Unix seconds. That
is, the integer number of seconds since 00:00:00 1st Jan 1970.

\subsection{Factory messages}
\label{sec:factorymessages}

The factory messages factory\_query and factory\_status are intended
for monitoring the state of the factories themselves, including
generic Linux health metrics such as free disk and CPU load. As well
as manual queries by administrators, these messages may also be
used for automated Nagios-style monitoring and alarms.

\subsubsection{factory\_query}
\label{sec:factoryquery}

The factory\_query message is sent to a factory to request a 
factory\_status message in response.

\begin{quote}
\begin{description}
\setlength{\parskip}{0pt}
\item[message\_type] ``factory\_query''
\item[vac\_version] Name and software version of Vac
\item[vacquery\_version] Name and version of the VacQuery protocol
\item[space] Vac space name
\item[cookie] Freely chosen by the sender
\end{description}
\end{quote}

\subsubsection{factory\_status}
\label{sec:factorystatus}

factory\_status messages are returned in response to 
factory\_query messages directed to a factory. They may
also be generated spontaneously and sent to a VacMon
service as described in section~\ref{sec:vacmon}.

The format and units of the disk and memory values are
aligned with the values returned by the relevant system
calls and the /proc interface.

\begin{quote}
\begin{description}
\setlength{\parskip}{0pt}
\item[message\_type] ``factory\_status''
\item[vac\_version] Name and software version of Vac
\item[vacquery\_version] Name and version of the VacQuery protocol
\item[cookie] Matching the value supplied by the recipient
\item[space] Vac space name
\item[factory] FQDN of the factory
\item[time\_sent] Time in Unix seconds
\item[site] Name of the site registered in the GOCDB, or the Vac space name if the site is not registered
\item[running\_cpus] Number of processors assigned to running VMs
\item[running\_machines] Number of running VMs
\item[running\_hs06] Total HS06 of running VMs
\item[max\_cpus] Maximum number of (logical) processors available to VMs
\item[max\_machines] Maximum possible number of VMs
\item[max\_hs06] Maximum HS06 available to all VMs
\item[boot\_time] The time when the factory booted up in Unix seconds
\item[factory\_heartbeat\_time] Time of the last heartbeat created by the VM factory agent in Unix secconds
\item[responder\_heartbeat\_time] Time of the last heartbeat created by the VacQuery responder service in Unix secconds
\item[mjf\_heartbeat\_time] Time of the last heartbeat created by the HTTP Machine/Job Features service in Unix secconds
\item[metadata\_heartbeat\_time] Time of the last heartbeat created by the HTTP Metadata service in Unix secconds
\item[vac\_disk\_avail\_kb] Free space available in Vac's workspace, in units of 1024 bytes
\item[root\_disk\_avail\_kb] Free space available on the root partition, in units of 1024 bytes
\item[vac\_disk\_avail\_inodes] Free inodes available in Vac's workspace
\item[root\_disk\_avail\_inodes] Free inodes available on the root partition
\item[load\_average] The 15 minute load average on the factory
\item[kernel\_version] The kernel version of the factory
\item[os\_issue] A string identifying the operating system (typically the first line of /etc/issue)
\item[swap\_used\_kb] Swap space in use on the factory, in units of 1024 bytes
\item[swap\_free\_kb] Free swap space, in units of 1024 bytes
\item[mem\_used\_kb] Physical memory in use on the factory, in units of 1024 bytes
\item[mem\_total\_kb] Free physical memory, in units of 1024 bytes
\end{description}
\end{quote}

\subsection{Machine messages}
\label{sec:machinemessages}

The machines\_query (plural) and machine\_status (singular) messages can
be used to create views of the VMs running within a Vac space, similar to
the views from the top command of running processes on a single host.

\subsubsection{machines\_query}
\label{sec:machinesquery}

The machines\_query message is sent to a factory to request a 
machine\_status message for each of its VM slots.

\begin{quote}
\begin{description}
\setlength{\parskip}{0pt}
\item[message\_type] ``machines\_query''
\item[vac\_version] Name and software version of Vac
\item[vacquery\_version] Name and version of the VacQuery protocol
\item[space] Vac space name
\item[cookie] Freely chosen by the sender
\end{description}
\end{quote}

\subsubsection{machine\_status}
\label{sec:machinestatus}

machine\_status messages are returned in response to machines\_query messages directed to a factory.

\begin{quote}
\begin{description}
\setlength{\parskip}{0pt}
\item[message\_type] ``machine\_status''
\item[daemon\_version] Name and software version of the factory daemon
\item[vacquery\_version] Name and version of the VacQuery protocol
\item[cookie] Matching the value supplied by the recipient
\item[space] Vac space name
\item[site] The site name, either as used by GOCDB or a string derived from the space name.
\item[factory] FQDN of the factory
\item[num\_machines] Number of machine\_status messages to expect from this factory
\item[time\_sent] Time in Unix seconds
\item[machine] Hostname of the VM slot
\item[state] State of the current or most recent VM in this slot, as a string
\item[uuid] Lowlevel UUID, as used by libvirtd
\item[created\_time] Unix time of the VM's creation
\item[started\_time] Unix time the VM entered the running state
\item[heartbeat\_time] Unix time when the VM was last observed to be running (this is not the same as any heartbeat generated within the VM)
\item[num\_processors] Number of logical or virtual processors assigned to this machine
\item[cpu\_seconds] CPU seconds used by the VM
\item[cpu\_percentage] Recent CPU percentage use. May be over 100\% for mulitprocessor VMs
\item[hs06] Total HEPSPEC06 for the processors assigned to this VM
\item[machinetype] Name of the machinetype
\item[shutdown\_message] Any shutdown message given by the last VM to run in this slot
\item[shutdown\_time] Unix time of the shutdown\_message
\item[fqan] Fully Qualified Attribute Name associate with the VM, typically identifying a VO and perhaps a group or role within it.
\end{description}
\end{quote}

\subsection{Machinetype messages}
\label{sec:machinetypemessages}

The machinetypes\_query (plural) and machinetype\_status (singular) messages 
are used by factories to gather information from neighbours within the
same Vac space about what they are running, and outcomes of recently started
VMs which have finished.

\subsubsection{machinetypes\_query}
\label{sec:machinetypesquery}

The machinetypes\_query message is sent to a factory to request a 
machinetype\_status message for each of the machinetypes it supports.

\begin{quote}
\begin{description}
\setlength{\parskip}{0pt}
\item[message\_type] ``machinetypes\_query''
\item[vac\_version] Name and software version of Vac
\item[vacquery\_version] Name and version of the VacQuery protocol
\item[space] Vac space name
\item[cookie] Freely chosen by the sender
\end{description}
\end{quote}

\subsubsection{machinetype\_status}
\label{sec:machinetypestatus}

machinetype\_status messages are returned in response to machinetypes\_query 
messages directed to a factory. They may
also be generated spontaneously and sent to a VacMon
service as described in section~\ref{sec:vacmon}.

\begin{quote}
\begin{description}
\setlength{\parskip}{0pt}
\item[message\_type] ``machinetype\_status''
\item[daemon\_version] Name and software version of the factory daemon
\item[vacquery\_version] Name and version of the VacQuery protocol
\item[cookie] Matching the value supplied by the recipient
\item[space] Vac space name
\item[site] The site name, either as used by GOCDB or a string derived from the space name.
\item[factory] FQDN of the factory
\item[num\_machinetypes] Number of machinetype\_status messages to expect from this factory
\item[time\_sent] Time in Unix seconds
\item[machinetype] Name of the machinetype
\item[running\_hs06] Total HEPSPEC06 of all the processors allocated to running VMs for this machinetype on this factory
\item[running\_machines] Number of running VMs for this machinetype on this factory
\item[running\_processors] Number of logical or virtual processors allocated to running VMs for this machinetype on this factory
\item[num\_before\_fizzle] Number of running VMs which have not yet reached fizzle\_seconds
\item[shutdown\_message] Shutdown message given by the most recently created VM for this machinetype on this factory which has finished
\item[shutdown\_time] Unix time of the shutdown\_message
\item[shutdown\_machine] Name of the VM slot associated with the shutdown\_message
\item[fqan] Fully Qualified Attribute Name associate with the VM, typically identifying a VO and perhaps a group or role within it.
\end{description}
\end{quote}

\subsection{VacMon services}
\label{sec:vacmon}

VacMon services receive factory\_status and machinetype\_status
messages from Vac daemons on UDP port 8884. These may be used for
Ganglia-style monitoring of individual sites or groups of sites.
As VacQuery messages are sent as JSON documents, they may be conveniently 
recorded in NoSQL data stores such as ElasticSearch.

\section{APEL}
\label{sec:apel}

VMLMs should support reporting of usage to the central APEL service
with messages of the type ``APEL-individual-job-message''. These are
the records used for conventional grid sites, rather than those developed
for cloud resources. 

VLMs must include the following in the messages:
\begin{quote}
\begin{description}
\setlength{\parskip}{0pt}
\item[FQAN] VOMS FQAN specified by the experiment when configuring the space
\item[SubmitHost] Must be of the form 
  \texttt{[space name] + "/" + [vmlm] + "-" + [VMLM host name]}, where
  ``vmlm'' is a lowercase name for the VMLM software such as ``vac''
\item[LocalJobId] VM UUID
\item[LocalUserId] VMLM hostname
\item[Queue] Name of the machinetype
\item[GlobalUserName] The space name converted to an X.500 DN with DC
  components. For example, vac01.example.com would become /DC=vac01/DC=example/DC=com
\item[InfrastructureDescription] Such as APEL-VAC or APEL-VCYCLE, with APEL and then
  an uppercase name for the VMLM software.
\item[Processors] The number of logical or virtual processors assigned to the VM.
\end{description}
\end{quote}

APEL Sync records must also be sent, and these can conveniently be
generated by each VMLM instance from an archive of the individual
job messages, as the SubmitHost is unique to the VMLM instance
in both cases.

In addition to grid-style APEL records generated by the VMLM, an underlying
cloud infrastracture may also be instrumented to submit cloud-style APEL 
usage records to the cental APEL service.
This is especially likely at sites participating in the EGI Federated Cloud.
In this case, resource providers must ensure that double counting is avoided
by disabling reporting from the VMLM to the central APEL service. 

\section{GOCDB}
\label{sec:gocdb}

Spaces should be registered in the GOCDB entries for the site, using 
appropriate service types. The service types \texttt{uk.ac.gridpp.vac} and 
\texttt{uk.ac.gridpp.vcycle} have been created for Vac and Vcycle
spaces.

Registration allows new Vacuum Platform resources to be discovered more easily
by experiments, and permits the declaration of downtimes for these services.

\section{Summary}
\label{sec:Summary}

This note has described the various interfaces between virtual machines and
virtual machine lifecycle managers required by the Vacuum Platform. Three new
interfaces (\$JOBOUTPUTS, VacQuery, and VacUserData) have been introduced,
and requirements set out for the use of Machine/Job Features, APEL, and
GOCDB with the platform. Procedures for how experiments providing VMs should
present their VM boot images and contextualization are also explained.

%\section*{Acknowledgments}

%\section*{References}

\begin{thebibliography}{9}
%% Use references in the format expected by JPCS (as used for CHEP proceedings)

%\bibitem{SHOULD-MUST} S. Bradner, RFC2119 ``Key words for use in RFCs to Indicate Requirement Levels'' (Internet Engineering Task Force)

\bibitem{HSFMJF} M.~Alef et al, HSF-TN-2016-02 ``Machine/Job Features Specification'' (HEP Software Foundation)

\bibitem{HSFVACPLAT1} A.~McNab, HSF-TN-2016-04 ``The Vacuum Platform'' (HEP Software Foundation)

\end{thebibliography}


%\addcontentsline{toc}{section}{References}

\end{document}
