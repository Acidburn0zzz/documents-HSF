\documentclass[12pt,a4paper]{article}

% Variables that controls behaviour
\usepackage{ifthen} % for conditional statements
\newboolean{pdflatex}
\setboolean{pdflatex}{true} % False for eps figures 

\newboolean{inbibliography}
\setboolean{inbibliography}{false} %True once you enter the bibliography


\textheight=230mm
\textwidth=160mm
\oddsidemargin=7mm
\evensidemargin=-10mm
\topmargin=-10mm
\headsep=20mm
\columnsep=5mm
\addtolength{\belowcaptionskip}{0.5em}

\renewcommand{\textfraction}{0.01}
\renewcommand{\floatpagefraction}{0.99}
\renewcommand{\topfraction}{0.9}
\renewcommand{\bottomfraction}{0.9}

\setlength{\hoffset}{-2cm}
\setlength{\voffset}{-2cm}

% Page defaults ...
\topmargin=0.5cm
\oddsidemargin=2.5cm
\textwidth=16cm
\textheight=22cm

% Don't chase after perfection
\raggedbottom
\sloppy

\usepackage{microtype}
\usepackage{lineno}    % Line numbering during drafting
\usepackage{xspace}    % Avoid problems with missing or double spaces after predefined symbold
\usepackage{caption}   % These three command get the figure and table captions automatically small
\renewcommand{\captionfont}{\small}
\renewcommand{\captionlabelfont}{\small}

%% Graphics
\usepackage{graphicx}  % to include figures (can also use other packages)
\usepackage{color}
\usepackage{colortbl}

%% Math
\usepackage{amsmath} % Adds a large collection of math symbols
\usepackage{amssymb}
\usepackage{amsfonts}
\usepackage{upgreek} % Adds in support for greek letters in roman typeset

%% fix to allow peaceful coexistence of line numbering and
%% mathematical objects
%% http://www.latex-community.org/forum/viewtopic.php?f=5&t=163
%%
\newcommand*\patchAmsMathEnvironmentForLineno[1]{%
\expandafter\let\csname old#1\expandafter\endcsname\csname #1\endcsname
\expandafter\let\csname oldend#1\expandafter\endcsname\csname
end#1\endcsname
 \renewenvironment{#1}%
   {\linenomath\csname old#1\endcsname}%
   {\csname oldend#1\endcsname\endlinenomath}%
}
\newcommand*\patchBothAmsMathEnvironmentsForLineno[1]{%
  \patchAmsMathEnvironmentForLineno{#1}%
  \patchAmsMathEnvironmentForLineno{#1*}%
}
\AtBeginDocument{%
\patchBothAmsMathEnvironmentsForLineno{equation}%
\patchBothAmsMathEnvironmentsForLineno{align}%
\patchBothAmsMathEnvironmentsForLineno{flalign}%
\patchBothAmsMathEnvironmentsForLineno{alignat}%
\patchBothAmsMathEnvironmentsForLineno{gather}%
\patchBothAmsMathEnvironmentsForLineno{multline}%
\patchBothAmsMathEnvironmentsForLineno{eqnarray}%
}

% Get hyperlinks to captions and in references.
% These do not work with revtex. Use "hypertext" as class option instead.
\usepackage{hyperref}    % Hyperlinks in references
\usepackage[all]{hypcap} % Internal hyperlinks to floats.

% Make this the last packages you include before the \begin{document}
\usepackage{cite} % Allows for ranges in citations
\usepackage{mciteplus}

\usepackage{longtable} % only for template

\begin{document}

\renewcommand{\thefootnote}{\fnsymbol{footnote}}
\setcounter{footnote}{1}

\begin{titlepage}
\pagenumbering{roman}


\vspace*{-1.5cm}
\centerline{\large THE HEP SOFTWARE FOUNDATION (HSF)}
\vspace*{1.5cm}
\noindent
\begin{tabular*}{\linewidth}{lc@{\extracolsep{\fill}}r@{\extracolsep{0pt}}}

\\
 & & HSF-TN-2015-01 \\  % ID 
 & & 10.5281/zenodo.1469623 \\ % DOI
 & & September 30, 2015 \\ % Date - Can also hardwire e.g.: 23 March 2010
 & & \\
% not in paper \hline
\end{tabular*}

\vspace*{4.0cm}

% Title --------------------------------------------------
{\bf\boldmath\huge
\begin{center}
  HSF Technical Notes policy
\end{center}
}

\vspace*{2.0cm}

% Authors -------------------------------------------------
\begin{center}
A.~McNab$^1$
\bigskip\\
{\it\footnotesize
$ ^1$University of Manchester
}
\end{center}

\vspace{\fill}

% Abstract -----------------------------------------------
\begin{abstract}
  \noindent
  The note describes the HSF Technical Notes policy, the rationale behind
  the notes series, and further recommendations. 
  This is the first version of the policy for the HEP
  Software Foundation
  (HSF) Technical Notes series, and itself serves as an example technical note.

\end{abstract}

\vspace*{2.0cm}

\vspace{\fill}

{\footnotesize 
\centerline{\copyright~Named authors on behalf of the HSF, licence \href{http://creativecommons.org/licenses/by/4.0/}{CC-BY-4.0}.}}
\vspace*{2mm}

\end{titlepage}

% empty page may follow the title page in long documents
%\newpage
%\setcounter{page}{2}
%\mbox{~}

\cleardoublepage

\renewcommand{\thefootnote}{\arabic{footnote}}
\setcounter{footnote}{0}

%%%% Uncomment next 2 lines if desired
%\tableofcontents
%\cleardoublepage

\pagestyle{plain} % restore page numbers for the main text
\setcounter{page}{1}
\pagenumbering{arabic}

%% Uncomment during drafting and review.
%% Comment before a final submission.
% \linenumbers

\section{Introduction}
\label{sec:Introduction}

This is the first version of the policy for the HEP Software Foundation
(HSF) Technical Notes series, and itself serves as an example technical note.

The technical notes series provides an early example of community level co-operation and 
interoperation that can be fostered by HSF.
It is hoped that groups of projects will co-operate on writing notes for
interfaces they provide or use, and HSF will offer support and publicity for
any meetings and workshops that result from these collaborative efforts.

The notes series provides archived, numbered documents that can be referenced in
software documentation and papers, and the notes themselves may also become the 
basis for papers associated with HSF member projects.

\section{Policies}
\label{sec:Policies}

\begin{enumerate}

\item HSF technical notes are informational, and they do not
constitute standards or recommendations. 

\item If a note needs to describe the required and optional elements of an interface between systems,
then use of the conventional key words 
``MUST'', 
``MUST NOT'', 
``REQUIRED'', 
``SHALL'', 
``SHALL NOT'', 
``SHOULD'', 
``SHOULD NOT'', 
``RECOMMENDED'', 
``MAY'', and      
``OPTIONAL''
as described in RFC 2119\cite{SHOULD-MUST} is preferred.

\item HSF limits its editorial input to checking that the document is within the
scope of HSF, and has the required format, grammar, and spelling.

\item HSF avoids attempting to judge whether the subject matter of a proposed technical 
note is sufficiently significant or not. Acceptance of a technical note
does not constitute an endorsement of its contents or importance.

\item A simple format for notes will be provided as a LaTeX template, but technical
notes must be supplied by authors as PDF files. Any method for creating a PDF file in
a similar format is acceptable, such as Word or Pages as a subsitute for 
LaTeX.

\item The copyright of a technical note remains with its authors, but the note
must be made available with a Creative Commons Attribution license, 
CC-BY-4.0 or later.

\item Notes which have been produced as reports for other purposes may be 
accepted in PDF form as HSF technical notes provided the title page of the
submitted PDF file conforms
to the HSF format including the assigned HSF identifier, and the entire document
is made available under the above Creative Commons license.

\item An archive of technical notes will be made available on a website
managed by HSF.

\item Each technical note will be assigned an identifier of the format
HSF-TN-YYYY-NN when it is finally accepted and archived by HSF. Notes should
be referenced as 
{\it Author(s), HSF-TN-YYYY-NN ``The Title'' (HEP Software
Foundation) } rather than with a URL.

\item Once published, substantive revisions to the technical note will not 
be accepted. This is similar to the IETF RFC model\cite{IETF-RFC}. 

\item Subsequent notes may be accepted
% from the same authors 
covering updates to the same subject matter 
% after one year, 
and a new note identifier will be issued. Projects are encouraged
to give version numbers to their interfaces to clarify the process of
documenting their interfaces as they evolve over time. HSF may prefix copies 
of older notes presented on the archive website with 
forward references to newer versions.

\end{enumerate}

\section{Procedure}
\label{sec:Procedure}

Draft technical notes should be submitted following the process published in
the Technical Notes section of the HSF website. 
The acceptance process will be managed by editors for the HSF Technical 
Notes series appointed by the HSF governing body (initially, by the HSF
interim Foundation Board). The series editors may co-opt
other people to act as additional editors for a given technical note.
The editorial process will include at least one week of consultation
on the proposed final draft within the HSF community.

\section{Recommendations}
\label{sec:Recommendations}

HSF working groups are asked to work towards producing technical notes which
record their outcomes.

Interested projects or groups of projects are encouraged to suggest 
technical notes documenting their interfaces and methods. It is likely that
many projects have documents like this already (e.g. as wiki pages, sections
of published papers or PhD theses) that can readily be reworked and
perhaps expanded into technical notes. 

\section{Standards}
\label{sec:Standards}

If there is a need for HSF
to manage standards, then a procedure will be developed for endorsing
suitable technical notes as standards documents.

\section{Summary}
\label{sec:Summary}

This technical note sets out the rationale, policies, and recommendations
for further HSF technical notes.

%\section*{Acknowledgments}

%\section*{References}

\begin{thebibliography}{9}
% Use references in the format expected by JPCS (as used for CHEP proceedings)

\bibitem{SHOULD-MUST} S. Bradner, RFC2119 ``Key words for use in RFCs to Indicate Requirement Levels'' (Internet Engineering Task Force)

\bibitem{IETF-RFC} S. Bradner, RFC2026 ``The Internet Standards Process -- Revision 3'' (Internet Engineering Task Force)

\end{thebibliography}


%\addcontentsline{toc}{section}{References}

\end{document}
