\documentclass[12pt,a4paper]{article}

% Variables that controls behaviour
\usepackage{ifthen} % for conditional statements
\newboolean{pdflatex}
\setboolean{pdflatex}{true} % False for eps figures 

\newboolean{inbibliography}
\setboolean{inbibliography}{false} %True once you enter the bibliography


\textheight=230mm
\textwidth=160mm
\oddsidemargin=7mm
\evensidemargin=-10mm
\topmargin=-10mm
\headsep=20mm
\columnsep=5mm
\addtolength{\belowcaptionskip}{0.5em}

\renewcommand{\textfraction}{0.01}
\renewcommand{\floatpagefraction}{0.99}
\renewcommand{\topfraction}{0.9}
\renewcommand{\bottomfraction}{0.9}

\setlength{\hoffset}{-2cm}
\setlength{\voffset}{-2cm}

% Page defaults ...
\topmargin=0.5cm
\oddsidemargin=2.5cm
\textwidth=16cm
\textheight=22cm

% Don't chase after perfection
\raggedbottom
\sloppy

\usepackage{microtype}
\usepackage{lineno}    % Line numbering during drafting
\usepackage{xspace}    % Avoid problems with missing or double spaces after predefined symbold
\usepackage{caption}   % These three command get the figure and table captions automatically small
\renewcommand{\captionfont}{\small}
\renewcommand{\captionlabelfont}{\small}

%% Graphics
\usepackage{graphicx}  % to include figures (can also use other packages)
\usepackage{color}
\usepackage{colortbl}

%% Math
\usepackage{amsmath} % Adds a large collection of math symbols
\usepackage{amssymb}
\usepackage{amsfonts}
\usepackage{upgreek} % Adds in support for greek letters in roman typeset

%% fix to allow peaceful coexistence of line numbering and
%% mathematical objects
%% http://www.latex-community.org/forum/viewtopic.php?f=5&t=163
%%
\newcommand*\patchAmsMathEnvironmentForLineno[1]{%
\expandafter\let\csname old#1\expandafter\endcsname\csname #1\endcsname
\expandafter\let\csname oldend#1\expandafter\endcsname\csname
end#1\endcsname
 \renewenvironment{#1}%
   {\linenomath\csname old#1\endcsname}%
   {\csname oldend#1\endcsname\endlinenomath}%
}
\newcommand*\patchBothAmsMathEnvironmentsForLineno[1]{%
  \patchAmsMathEnvironmentForLineno{#1}%
  \patchAmsMathEnvironmentForLineno{#1*}%
}
\AtBeginDocument{%
\patchBothAmsMathEnvironmentsForLineno{equation}%
\patchBothAmsMathEnvironmentsForLineno{align}%
\patchBothAmsMathEnvironmentsForLineno{flalign}%
\patchBothAmsMathEnvironmentsForLineno{alignat}%
\patchBothAmsMathEnvironmentsForLineno{gather}%
\patchBothAmsMathEnvironmentsForLineno{multline}%
\patchBothAmsMathEnvironmentsForLineno{eqnarray}%
}

% Get hyperlinks to captions and in references.
% These do not work with revtex. Use "hypertext" as class option instead.
\usepackage{hyperref}    % Hyperlinks in references
\usepackage[all]{hypcap} % Internal hyperlinks to floats.

% Make this the last packages you include before the \begin{document}
\usepackage{cite} % Allows for ranges in citations
\usepackage{mciteplus}

\usepackage{longtable} % only for template

\begin{document}

\renewcommand{\thefootnote}{\fnsymbol{footnote}}
\setcounter{footnote}{1}

\begin{titlepage}
\pagenumbering{roman}


\vspace*{-1.5cm}
\centerline{\large THE HEP SOFTWARE FOUNDATION (HSF)}
\vspace*{1.5cm}
\noindent
\begin{tabular*}{\linewidth}{lc@{\extracolsep{\fill}}r@{\extracolsep{0pt}}}

\\
 & & HSF-TN-2016-MJF \\  % ID 
 & & \today \\ % Date - Can also hardwire e.g.: 23 March 2010
 & & \\
% not in paper \hline
\end{tabular*}

\vspace*{4.0cm}

% Title --------------------------------------------------
{\bf\boldmath\huge
\begin{center}
  Machine/Job Features
\end{center}
}

\vspace*{2.0cm}

% Authors -------------------------------------------------
\begin{center}
M.~Alef$^1$,
T.~Cass$^2$,
J.J.~Keijser$^3$,
A.~McNab$^4$,
S.~Roiser$^2$,
U.~Schwickerath$^2$,
I.~Sfiligoi$^5$
\bigskip\\
{\it\footnotesize
$ ^1$Karlsruhe Institute of Technology \\
$ ^2$CERN \\
$ ^3$NIKHEF \\
$ ^4$University of Manchester \\
$ ^5$Fermi National Accelerator Laboratory \\
}
\end{center}

\vspace{\fill}

% Abstract -----------------------------------------------
\begin{abstract}
  \noindent

Within the HEPiX virtualization group and the 
WLCG MJF Task Force, a mechanism 
has been developed which provides
access to detailed information about the current host and the current job to
the job itself. This allows user
payloads to access meta information, independent of the current batch
system or virtual machine model. This
information includes the performance of the node and the
remaining run time for the current job.

\end{abstract}

\vspace*{2.0cm}

\vspace{\fill}

{\footnotesize 
\centerline{\copyright~Named authors on behalf of the HSF, licence \href{http://creativecommons.org/licenses/by/4.0/}{CC-BY-4.0}.}}
\vspace*{2mm}

\end{titlepage}

% empty page may follow the title page in long documents
%\newpage
%\setcounter{page}{2}
%\mbox{~}

\cleardoublepage

\renewcommand{\thefootnote}{\arabic{footnote}}
\setcounter{footnote}{0}

%%%% Uncomment next 2 lines if desired
%\tableofcontents
%\cleardoublepage

\pagestyle{plain} % restore page numbers for the main text
\setcounter{page}{1}
\pagenumbering{arabic}

%% Uncomment during drafting and review.
%% Comment before a final submission.
%\linenumbers

\section{Introduction}
\label{sec:Introduction}

Within the HEPiX virtualization group\cite{HEPIXMJF} and the 
WLCG MJF Task Force\cite{WLCGMJF}, a mechanism 
has been developed which provides
access to detailed information about the current host and the current job to
the job itself. This allows user
payloads to access meta information, independent of the current batch
system or virtual machine model. 

The proposed schema is made to be extensible so that 
additional information can be added. The purpose of this note is to define the
specifications and use case of this schema. It should be seen as the source
of information for the actual implementation of the scripts required by the
sites to provide it.

\section{Aims}
\label{sec:Aims}

\begin{itemize}
\item The proposed schema must be unique and leave no room for interpretation of the values provided.
      For this reason, basic information is used which is well defined across sites.
\item Host and job information can be both static (like the HS06\cite{HEPIXHS06} rating of the 
      hardware) and dynamic (eg shutdown time may be set at any time by the site.)
\item Job specific files will be readable and possibly owned by the user and
      residing on a /tmp like area
\item The implementation, that is the creation of the files and their contents, can
      be highly site specific. A sample implementation can be done per batch
      system in use, but it is understood that sites are allowed to change the
      implementation, provided that the created numbers match the definitions
      given in this note.
\end{itemize}

\section{Use cases}
\label{sec:Use cases}

The use cases considered in developing the protocols included:

\begin{enumerate}
\item The job needs to calculate the remaining time it is allowed to run.
\item The job needs to know how long it was already running.
\item The job wants to know the performance of the processors allocated to it in order to calculate the remaining time it will need to complete (for CPU intensive jobs).
\item A host needs to be drained, and the payload needs to be informed of the planned shutdown time.
\item A multiprocessor user job on a non-exclusive node needs to know how many threads or processes it is allowed to start. This
      especially useful in a late-binding scenario where the pilot reserved the processors and the user payload needs to discover this.
\item A user job wants to know how many processors are allocated to the current job.
\item A user job wants to know the maximum amount of scratch disk it is allowed to use.
\item A user job wants to set up memory limits to protect itself from being killed by the batch system automatically.
\end{enumerate}

\section{Definitions}
\label{sec:Definitions}

On VM-based systems, references to ``jobs'' are to be interpreted
as ``virtual machines'' and ``machines'' as ``hypervisors''.

When jobs are running within virtual machines, the entity that provides 
the system level configuration or contextualization of the VM acts
as the resource provider referred to in the rest of this note.

\section{Environment variables}
\label{sec:EnvironmentVariables}

For each job, two environment variables may be set, with the names
\$MACHINEFEATURES and \$JOBFEATURES.

These environment variables are the base interface for the user payload.
Their values must be provided for the job by the resource provider. 

In the
case of virtual machines on IaaS cloud platforms, the virtual machine
may discover the values to set for the environment variables from 
``machinefeatures'' and ``jobfeatures'' metadata keys provided by resource
provider via the cloud
infrastructure. These metadata keys should only be accessed once in the
lifetime of each virtual machine. Alternatively, the values to set may
be supplied as part of the contextualization of the virtual machines.

\section{Directories}
\label{sec:Directories}

The environment variables point to directories created by the resource
provider. Inside, the file name is the key, the contents are the values, so
that files can be referred to with expressions like
\$MACHINEFEATURES/shutdowntime . The directory name should not include the
trailing slash. These directories are either local directories in the
filesystem or sections of the URL space on an HTTP(S) server. The user
positively determines whether the files are to be opened locally or over
HTTP(S) by checking for a leading slash or the prefix http:// or https://
respectively. Typically this can achieved using library functions which can
transparently handle local files and remote URLs when opening files. 

Unlike metadata keys, the key/value files may be
accessed multiple times to check for changes in value or in the
absence of caching by the user. An HTTP(S) server may provide HTTP cache
control and expiration information which the user may use to reduce the
number of queries.
All files in the directories must be readable by both the user and the
resource provider services, and have file names which only consist of
lowercase letters, numbers, and underscores.

\section{\$MACHINEFEATURES}
\label{sec:MACHINEFEATURES}

Host-specific key/value pairs which are all:

\begin{itemize}
\item Found in the directory pointed to by \$MACHINEFEATURES
\item Readable by the user who is executing the original job. In the case
      of pilots this would be the pilot user at the site.
\item Required unless the resource provider cannot determine their value.
\item Static unless otherwise stated.
\end{itemize}

\begin{description}

\item[total\_cpu] Number of processors which may be allocated to
jobs. Typically the number of processors seen by the operating system on one
worker node (that is the number of ``processor :'' lines in /proc/cpuinfo on
Linux), but potentially set to more or less than this for performance
reasons. (Use case 3.)

\item[hs06] Total HS06 rating of the full machine in its current
setup. HS06 is measured following the HEPiX recommendations\cite{HEPIXHS06}, with 
HS06 benchmarks run in parallel, one for each processor which may be 
allocated to jobs. (Use case 3.)

\item[shutdowntime] Shutdown time for the machine as a UNIX time stamp 
in seconds. The value is dynamic and optional. If the file is missing, 
no shutdown is foreseen.  (Use case 4.)

\item[grace\_secs] If the resource provider announces a
shutdown time to the jobs on this host, that time will not be less than 
grace\_secs seconds after the moment the shutdown time is set. This 
allows jobs to begin packages of work knowing that there will be sufficient 
time for them to be completed even if a shutdown time is announced. 
This value is required if a shutdown time will be set or changed
which will affect any jobs which have already started on this host.

\end{description}

\section{\$JOBFEATURES}
\label{sec:JOBFEATURES}

Job specific key/value pairs which are all:

\begin{itemize}
\item Found in the directory pointed to by \$JOBFEATURES
\item Readable and possibly owned by the user who is executing the original
      job. In the case of pilots this would be the pilot user at the site.
\item Required unless the resource provider cannot determine their value.
\item Created before the job starts and static unless otherwise stated, or 
      unless the batch system has a recognised way of changing the
      parameters of the job in a way the job is guaranteed to be aware
      of. For example, if there is a mechanism for a job to
      release processors, then the resource provider may update
      allocated\_cpu when this happens.
\end{itemize}

\begin{description}

\item[allocated\_cpu] Number of processors allocated to the current job. 
(Use case~5.) 

\item[hs06\_job] Total HS06 rating for the processors allocated
to this job. The job's share is calculated by the resource provider from 
per-processor HS06 measurements made for the machine. (Use case~3.)

\item[shutdowntime\_job] Dynamic value. Shutdown time as a UNIX time 
stamp in seconds. If the file is missing no job shutdown
is foreseen. The job needs to have finished all of its processing when
the shutdown time has arrived. (Use case~1.)

\item[grace\_secs\_job] If the resource provider announces a
shutdowntime\_job to the job, it will not be less than grace\_secs\_job
seconds after the moment the shutdown time is set. This allows jobs to
begin packages of work knowing that there will be sufficient time for
them to be completed even if a shutdown time is announced.
This value is static and required if a shutdown time will be set or changed
after the job has started.

\item[jobstart\_secs] UNIX time stamp in seconds of the time when the job 
started on the worker node. For a pilot job scenario, this is when the 
batch system started the pilot job, not when the user payload started to 
run. (Use case~2.)

\item[job\_id] A string of printable non-whitespace ASCII characters
used by the resource provider to
identify the job at the site. In batch environments, this should simply be
the job ID. In virtualized environments, job\_id will typically contain the
UUID of the VM.

\item[wall\_limit\_secs] Elapsed time limit in seconds, starting from
jobstart\_secs. This is not scaled up for multiprocessor jobs.
(Use case~1.)

\item[cpu\_limit\_secs] CPU time limit in seconds. For multiprocessor jobs
this is the total for all processes started by the job. (Use case~1.)

\item[max\_rss\_bytes] Resident memory usage limit, if any, in bytes
for all processes started by this job. (Use case~8.) 

\item[max\_swap\_bytes] Swap limit, if any, in bytes for all processes
started by this job. (Use case~8) 

\item[scratch\_limit\_bytes] Scratch space limit if any. If no quotas
are used on a shared system, this corresponds to the full scratch 
space available to all jobs which run on the host. User jobs from
EGI-registered VOs expect the ``max size of scratch space used by jobs''
value on their VO ID Card\cite{VOIDCARD} to be available to each job in the worst case.
If there is a recognised procedure for informing the job of the location
of the scratch space (eg EGI's \$TMPDIR policy\cite{EGITMPDIR}), then this value refers to that
space. (Use case~7.) 

\end{description}

\clearpage

\section{Summary}
\label{sec:Summary}

This note describes how the \$MACHINEFEATURES and \$JOBFEATURES variables
may be set and used by jobs to obtain meta information from resource
providers in a uniform way across different batch and virtual machine
systems.
\vspace{0.5cm}

The following key/value pairs have been defined:

\begin{center}
\begin{tabular}{|l|l|p{5cm}}
\hline
\$MACHINEFEATURES & \$JOBFEATURES \\
\hline
total\_cpu & allocated\_cpu \\
hs06 & hs06\_job \\
shutdowntime & shutdowntime\_job \\
grace\_secs & grace\_secs\_job \\
 & jobstart\_secs \\
 & job\_id \\
 & wall\_limit\_secs \\
 & cpu\_limit\_secs \\
 & max\_rss\_bytes \\
 & max\_swap\_bytes \\
 & scratch\_limit\_bytes \\
\hline
\end{tabular}
\end{center}


%\section*{Acknowledgments}

%\section*{References}

\begin{thebibliography}{9}
% Use references in the format expected by JPCS (as used for CHEP proceedings)

\bibitem{HEPIXMJF} T. Cass, ``Environmental Information on WN'', Grid
Deployment Board, CERN, 13 June 2012, retrieved from
https://indico.cern.ch/event/155069/

\bibitem{WLCGMJF} ``Machine / Job Features Task Force'', \\
 https://twiki.cern.ch/twiki/bin/view/LCG/MachineJobFeatures

\bibitem{HEPIXHS06} ``HEP-SPEC06 Benchmark'', https://w3.hepix.org/benchmarks/

\bibitem{VOIDCARD} ``The VO ID Card system'', http://operations-portal.egi.eu/vo/help

\bibitem{EGITMPDIR} P. Solagna, ``EGI policy for the TMPDIR environment variable usage'',
 EGI Document 1119, retrieved from https://documents.egi.eu/public/ShowDocument?docid=1119

%\bibitem{SHOULD-MUST} S. Bradner, RFC2119 ``Key words for use in RFCs to Indicate Requirement Levels'' (Internet Engineering Task Force)

%\bibitem{IETF-RFC} S. Bradner, RFC2026 ``The Internet Standards Process -- Revision 3'' (Internet Engineering Task Force)

\end{thebibliography}


%\addcontentsline{toc}{section}{References}

\end{document}
